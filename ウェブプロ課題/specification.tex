% !TEX program = uplatex
% コンパイル例:
%   uplatex specification.tex
%   dvipdfmx specification.dvi

\documentclass[fontsize=11pt]{jlreq}

\usepackage{graphicx}
\usepackage{url}
\usepackage{booktabs}
\usepackage{longtable}
\usepackage{geometry}
\geometry{margin=25mm}

\title{Webアプリケーション仕様書\\(F1 2026年出場チーム,歴代火影,W杯2022ベスト8)}
\author{作成者:\underline{\hspace{50mm}}}
\date{\today}

\begin{document}
\maketitle

\section{概要}
本書は,Express と EJS を用いて構築したWebアプリケーションの仕様をまとめたものである.
対象データは,「F1 2026年出場チーム」「歴代火影」「W杯2022ベスト8」の3種類とし,各データセットに対してCRUD(Create,Read,Update,Delete)を提供する.

本アプリは教育目的の簡易実装であり,データ保存はサーバのメモリ上で行う.そのため,サーバ再起動によりデータは初期状態へ戻る.

\section{動作環境}
\begin{tabular}{ll}
\toprule
項目 & 内容 \\
\midrule
OS & Windows / macOS(Node.js が動作する環境) \\
言語処理系 & Node.js(LTS推奨) \\
フレームワーク & Express \\
テンプレート & EJS \\
ポート & 8080 \\
\bottomrule
\end{tabular}

\section{起動手順}
プロジェクト直下で次を実行する.

\begin{tabular}{ll}
\toprule
手順 & コマンド \\
\midrule
依存関係の導入 & \texttt{npm install} \\
起動 & \texttt{npm start} \\
\bottomrule
\end{tabular}

起動後,ブラウザで \texttt{http://localhost:8080/} にアクセスする.

\part{利用者向け仕様}

\section{利用者の想定}
利用者は,データの閲覧を主目的として本アプリを利用する.
本仕様の「利用者」は,データの内容を参照し,必要に応じて詳細ページを確認する.

\section{画面一覧}
\begin{longtable}{ll}
\toprule
画面名 & URL \\
\midrule
トップ & \texttt{/} \\
F1 2026 一覧 & \texttt{/f1\_2026} \\
F1 2026 詳細 & \texttt{/f1\_2026/\{number\}} \\
歴代火影 一覧 & \texttt{/hokage} \\
歴代火影 詳細 & \texttt{/hokage/\{number\}} \\
W杯2022 ベスト8 一覧 & \texttt{/wc2022\_best8} \\
W杯2022 ベスト8 詳細 & \texttt{/wc2022\_best8/\{number\}} \\
\bottomrule
\end{longtable}

\section{基本操作}
\subsection{一覧表示}
トップ画面から任意のデータセットを選択すると,一覧画面に遷移する.一覧画面では,各行の名称をクリックすることで詳細画面に遷移できる.

\subsection{詳細表示}
詳細画面では,選択した要素の各項目を表形式で表示する.

\part{管理者向け仕様}

\section{管理者の想定}
管理者は,データの追加・編集・削除を行い,Web上に表示される内容を更新する.
教育用の簡易アプリのため,認証機能は実装しない.実運用では,管理者向け機能を認証で保護することが望ましい.

\section{管理機能}
\subsection{追加(Create)}
各一覧画面の「追加」リンクを選択すると,追加フォーム(静的HTML)へ遷移する.送信後,対象一覧へ戻り,追加内容が反映される.

\subsection{編集(Update)}
各一覧画面の「編集」リンクを選択すると,編集画面へ遷移する.編集画面から更新を送信すると,対象一覧へ戻り,変更が反映される.

\subsection{削除(Delete)}
各一覧画面および詳細画面の「削除」リンクを選択すると,対象の行が削除され,対象一覧へ戻る.削除時はブラウザの確認ダイアログを表示する.

\section{入力項目と制約}
\subsection{F1 2026年出場チーム}
\begin{tabular}{lll}
\toprule
項目 & 必須 & 制約 \\
\midrule
チーム名(name) & 必須 & 80文字以内 \\
メモ(note) & 任意 & 制限なし(画面表示上は短文推奨) \\
\bottomrule
\end{tabular}

\subsection{歴代火影}
\begin{tabular}{lll}
\toprule
項目 & 必須 & 制約 \\
\midrule
名称(name) & 必須 & 80文字以内 \\
Roman(roman) & 任意 & 画面表示上は短文推奨 \\
メモ(note) & 任意 & 画面表示上は短文推奨 \\
\bottomrule
\end{tabular}

\subsection{W杯2022 ベスト8}
\begin{tabular}{lll}
\toprule
項目 & 必須 & 制約 \\
\midrule
国・地域(name) & 必須 & 80文字以内 \\
\bottomrule
\end{tabular}

\section{データの永続化について}
本アプリはメモリ上の配列で管理するため,サーバ再起動によりデータは初期化される.
管理者が更新内容を残す必要がある場合,JSON API を利用して手動でバックアップを取る,あるいはDBを導入する拡張を行う.

\part{開発者向け仕様}

\section{システム構成}
\subsection{ディレクトリ構成}
\begin{verbatim}
webpro_f1_hokage_wc2022/
  app.js
  package.json
  views/
    index.ejs
    f1_2026.ejs
    f1_2026_detail.ejs
    f1_2026_edit.ejs
    hokage.ejs
    hokage_detail.ejs
    hokage_edit.ejs
    wc2022_best8.ejs
    wc2022_best8_detail.ejs
    wc2022_best8_edit.ejs
  public/
    style.css
    f1_2026_new.html
    hokage_new.html
    wc2022_best8_new.html
  README.md
  specification.tex
\end{verbatim}

\subsection{サーバ側の設計}
\texttt{app.js} は,以下の役割を持つ.

\begin{tabular}{ll}
\toprule
役割 & 内容 \\
\midrule
ルーティング & 各URLに対して,一覧・詳細・編集などの処理を割り当てる \\
テンプレート描画 & \texttt{res.render()} により EJS を描画する \\
フォーム受信 & \texttt{express.urlencoded()} により \texttt{POST} を扱う \\
簡易バリデーション & 必須入力と文字数の最小チェックを行う \\
\bottomrule
\end{tabular}

\section{ルーティング仕様}
\subsection{F1 2026年出場チーム}
\begin{longtable}{lll}
\toprule
HTTP & URL & 処理 \\
\midrule
GET & \texttt{/f1\_2026} & 一覧表示 \\
GET & \texttt{/f1\_2026/create} & 追加フォームへリダイレクト \\
POST & \texttt{/f1\_2026} & 追加登録 \\
GET & \texttt{/f1\_2026/\{number\}} & 詳細表示 \\
GET & \texttt{/f1\_2026/edit/\{number\}} & 編集画面 \\
POST & \texttt{/f1\_2026/update/\{number\}} & 更新 \\
GET & \texttt{/f1\_2026/delete/\{number\}} & 削除 \\
\bottomrule
\end{longtable}

\subsection{歴代火影}
\begin{longtable}{lll}
\toprule
HTTP & URL & 処理 \\
\midrule
GET & \texttt{/hokage} & 一覧表示 \\
GET & \texttt{/hokage/create} & 追加フォームへリダイレクト \\
POST & \texttt{/hokage} & 追加登録 \\
GET & \texttt{/hokage/\{number\}} & 詳細表示 \\
GET & \texttt{/hokage/edit/\{number\}} & 編集画面 \\
POST & \texttt{/hokage/update/\{number\}} & 更新 \\
GET & \texttt{/hokage/delete/\{number\}} & 削除 \\
\bottomrule
\end{longtable}

\subsection{W杯2022 ベスト8}
\begin{longtable}{lll}
\toprule
HTTP & URL & 処理 \\
\midrule
GET & \texttt{/wc2022\_best8} & 一覧表示 \\
GET & \texttt{/wc2022\_best8/create} & 追加フォームへリダイレクト \\
POST & \texttt{/wc2022\_best8} & 追加登録 \\
GET & \texttt{/wc2022\_best8/\{number\}} & 詳細表示 \\
GET & \texttt{/wc2022\_best8/edit/\{number\}} & 編集画面 \\
POST & \texttt{/wc2022\_best8/update/\{number\}} & 更新 \\
GET & \texttt{/wc2022\_best8/delete/\{number\}} & 削除 \\
\bottomrule
\end{longtable}

\section{データ設計}
本アプリでは,配列(JavaScriptのオブジェクト配列)としてデータを管理する.
\texttt{number} は配列の添字として利用し,\texttt{id} は表示用の識別子として保持する.
\texttt{id} は追加時に最大値へ1を加えることで生成する.

\section{エラー処理}
存在しない番号(添字)へのアクセス,および必須項目未入力はエラーとして扱い,HTTPステータスコードを返す.
本実装では簡易化のため,エラー画面のテンプレートは用意せず,\texttt{res.status(400/404).send(...)} を用いる.

\section{拡張案}
実運用を想定する場合,以下の拡張が有効である.

\begin{tabular}{ll}
\toprule
拡張案 & 内容 \\
\midrule
認証・認可 & 管理機能をログインで保護する \\
DB導入 & SQLite / MySQL などに永続化する \\
入力検証の強化 & 型チェック,範囲チェック,重複チェック \\
テンプレート共通化 & EJSの共通レイアウト化により重複を削減する \\
\bottomrule
\end{tabular}

\end{document}
